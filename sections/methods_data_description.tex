\subsection{Data Description and Notation}

We define the notation as follows: 

The input vector \hbox{\(\mathbf{x} = [x_0, x_1, x_2, \ldots, x_9]^T\)}, where \(x_1, x_2, \ldots, x_9\) are the nine features, and \(x_0 = 1\) is a bias term used to adjust the output.

The output \(y\) is defined differently depending on the model:

For Support Vector Machines (SVM):
\[
y = 
\begin{cases} 
1 & \text{if it is classified as malignant} \\
-1 & \text{if it is classified as benign}
\end{cases}
\]

For Multi-Layer Perceptrons (MLP):
\[
y = 
\begin{cases} 
1 & \text{if it is classified as malignant} \\
0 & \text{if it is classified as benign}
\end{cases}
\]

The weight vector for the input vector is denoted as \hbox{\(\mathbf{w}^{(\mathbf{x})} = [w_0, w_1, \ldots, w_9]^T\)}.
