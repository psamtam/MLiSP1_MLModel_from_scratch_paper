\subsection{Data Description and Notation}

The Breast Cancer Wisconsin (Original) dataset consists of 699 instances, each with 9 features. The total size of the dataset is \(699 \times 9\), with 16 missing values only in the \texttt{"Bare\_nuclei"} feature. As a result, the instances with missing values were removed. The box plots show that the benign cases have relatively small differences in values, while the differences in malignant cases are large. Furthermore, the KDE plots indicate that malignant cases have multimodal distribution, while the benign data are more concentrated.

The notation is defined as follows: 

The input vector \hbox{\(\mathbf{x} = [x_0, x_1, x_2, \ldots, x_9]^T\)}, where \(x_1, x_2, \ldots, x_9\) are the nine features, and \(x_0 = 1\) is a bias term used to adjust the output.

The output \(y\) is defined as follows:

\[
y = 
\begin{cases} 
1 & \text{(if it is classified as malignant)} \\
0 & \text{(if it is classified as benign, for MLP)} \\
-1 & \text{(if it is classified as benign, for SVM)}
\end{cases}
\]

The weight vector for the input vector is denoted as \hbox{\(\mathbf{w}^{(\mathbf{x})} = [w_0, w_1, \ldots, w_9]^T\)}.
