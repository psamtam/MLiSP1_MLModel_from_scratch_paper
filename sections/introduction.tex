\section{Introduction}

Breast cancer is one of the most common and life-threatening diseases in women around the world \cite{WHO2024}.  Early and accurate diagnosis is essential for improving patient outcomes \cite{Sun2017}. However, typical tumour diagnosis involves multiple stages \cite{Cardoso2019} and is both time-consuming and financially burdensome. Machine learning (ML) offers an alternative by automating tumour classification based on  extracted tumour characteristics. 

High-level programming interfaces for building machine learning (ML) models, such as scikit-learn \cite{scikit_learn} and TensorFlow \cite{TensorFlow2024}, are widely available. However, a deep understanding of the mathematical foundation behind the models is crucial for modifying, optimising, and interpreting the model algorithms. In this paper, ML models for tumour classification from scratch are developed without relying on any high-level ML libraries. Two fundamental models for classification: Support Vector Machines (SVM) and Multi-Layer Perceptron (MLP), are focused on. 

The Breast Cancer Wisconsin (Original) dataset is used for training and evaluation, and performance metrics such as accuracy, precision, recall (sensitivity), and specificity are computed. The rest of this paper is structured as follows: the Methods section describes the data and the implementation of SVM and MLP; the Result section demonstrates and compares the performance of both models; the Conclusion summarises the findings and suggests future research directions.